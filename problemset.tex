\documentclass[11pt]{article} 
\usepackage{amsmath,amsthm,amsfonts,amssymb,amscd}
\usepackage{multirow,booktabs}
\usepackage[table]{xcolor}
\usepackage{fullpage}
\usepackage{lastpage}
\usepackage{enumitem}
\usepackage{fancyhdr}
\usepackage{mathrsfs}
\usepackage{wrapfig}
\usepackage{setspace}
\usepackage{calc}
\usepackage{multicol}
\usepackage{cancel}
\usepackage[retainorgcmds]{IEEEtrantools} 
\usepackage[margin=3cm]{geometry}
\usepackage{amsmath}
\newlength{\tabcont}
\setlength{\parindent}{0.0in}
\setlength{\parskip}{0.05in} 
\usepackage{empheq} 
\usepackage{framed}
\usepackage[most]{tcolorbox}
\usepackage{xcolor}
\parindent 0in
\parskip 12pt
\geometry{margin=1in, headsep=0.25in}
\theoremstyle{definition} 
\newtheorem{defn}{Definition}
\newtheorem{reg}{Rule}
\newtheorem{exer}{Exercise}
\newtheorem{note}{Note} 
\newtheorem{solution}{Solution} 
\newtcolorbox{redbox}{colback=red!5!white,colframe=red!75!black}
\newtcolorbox{bluebox}{colback=blue!5!white,colframe=blue!75!black} 
\newtcbtheorem 
  [] 
  {definition} 
  {Definition} 
  { 
    colback=blue!5,
    colframe=blue!35!black,
    fonttitle=\bfseries,
  } 
  {def} 
\newtcbtheorem 
  [] 
  {theorem} 
  {Theorem} 
  { 
    colback=red!5,
    colframe=red!35!black,
    fonttitle=\bfseries,
  } 
  {def} 
\newtcbtheorem 
  [] 
  {problem} 
  {Problem} 
  { 
    colback=green!5,
    colframe=green!35!black,
    fonttitle=\bfseries,
  } 
  {def} 

\newcommand\ASolution{ 
  \stepcounter{solution} 
  \textbf{\textit{Solution}:  }\\ 
}

\begin{document} 

{\noindent\Huge\bf  \\[0.5\baselineskip] {\fontfamily{cmr}\selectfont  Problem Set 1}         }\\[2\baselineskip] % Title
{ {\bf \fontfamily{cmr}\selectfont Introduction to Algorithms}\\ {\textit{\fontfamily{cmr}\selectfont     October 21, 2019}}}~~~~~~~~~~~~~~~~~~~~~~~~~~~~~~~~~~~~~~~~~~~~~~~~~~~~~~~~~~~~~~~~~~~~~~~~~~~~~    {\large \textsc{Jinha Kim}} %\footnote{With Elizabeth Petrik}% } 
% \\[1.4\baselineskip] 

\hrulefill

\begin{problem}{Problem}{test}
      English surjective $X$ injective
      \[ a:\mathbb{N} \rightarrow X, \quad (a_1,a_2,a_3,...) = (a_n)_{n \geq 1} = (a_n)_n = (a_n) \]
      homeomorphic, telegraphic \dots
\end{problem} 

\begin{ASolution}  
  Hi $\dot{r_0} = \dot{R} + V$. 
  Newton's second law for the inertial reference frame by differentiate and multiplying by mass is:
  $F_{\text{inertial}} = -mA = -m\ddot{R}$
\begin{proof}
Thus, the proof is complete. 
\end{proof} 
\end{ASolution} 

\begin{problem}{Problem}{test}
  English surjective $X$ injective
  \[ a:\mathbb{N} \rightarrow X, \quad (a_1,a_2,a_3,...) = (a_n)_{n \geq 1} = (a_n)_n = (a_n) \]
  homeomorphic, telegraphic \dots
\end{problem} 

% \begin{bluebox}
\begin{ASolution}  
Hi $\dot{r_0} = \dot{R} + V$. 
Newton's second law for the inertial reference frame by differentiate and multiplying by mass is:
$F_{\text{inertial}} = -mA = -m\ddot{R}$
\begin{proof}
Thus, the proof is complete. 
\end{proof} 
\end{ASolution} 
% \end{bluebox} 

\end{document}  